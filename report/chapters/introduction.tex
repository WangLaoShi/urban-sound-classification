\section{Introduction}

\emph{I declare that this material, which I now submit for assessment, 
is entirely my own work and has not been taken from the work of others, 
save and to the extent that such work has been cited and acknowledged within 
the text of my work. I understand that plagiarism, collusion, and copying 
are grave and serious offences in the university and accept the penalties that 
would be imposed should I engage in plagiarism, collusion or copying. 
This assignment, or any part of it, has not been previously submitted by 
me or any other person for assessment on this or any other course of study.}

The goal of this project is to build a neural network to classify audio 
files from the \emph{UrbanSound8k} dataset~\cite{dataset}.

This dataset contains audio divided in ten classes, each one representing 
a different type of city sound, for instance, we can find \emph{car horns}, 
\emph{dogs barking}, \emph{sirens}, etc. A deeper discussion about 
the dataset is present in the Subsection \vref*{dataset-structure}.

The presented methodology is composed of three main parts.
The first step is to extract relevant features from audio files using the
\emph{Librosa} library~\cite{librosa}.  This is discussed on Section 
\vref*{feature-extraction}.\\
The next step consists in composing and refining a neural network 
to classify the data obtained from the previous step. This part is 
made possible by the \emph{Keras} library and it is discussed in 
Section \vref*{model-definition}~\cite{keras}.\\
Lastly, results from the classification, namely accuracy 
and standard deviation among test sets, are presented in 
Section \vref*{results}.

The project is developed in \emph{Python} and the code is structured in 
a \emph{src} package~\cite{python}. Each one of the sub-packages contains code to deal 
with the different parts of the project. For instance, the \emph{data} folder 
contains the classes to extract features and to manage a dataset.

In addition to the package there is a \emph{notebook} folder containing
the different steps of the project and the various experiments made.

\newpage