\section{Introduction}

\emph{I declare that this material, which I now submit for assessment, 
is entirely my own work and has not been taken from the work of others, 
save and to the extent that such work has been cited and acknowledged within 
the text of my work. I understand that plagiarism, collusion, and copying 
are grave and serious offences in the university and accept the penalties that 
would be imposed should I engage in plagiarism, collusion or copying. 
This assignment, or any part of it, has not been previously submitted by 
me or any other person for assessment on this or any other course of study.}

The goal of this project is to build a neural network to classify audio 
files from the \emph{UrbanSound8k} dataset~\cite{dataset}.

This dataset contains audio divided in ten classes, each one representing 
a different type of city sound, for instance, we can find \emph{car horns}, 
\emph{dogs barking}, \emph{sirens}, etc. A deeper discussion about 
the dataset is made at Subsection \vref*{dataset-structure}.

The presented methodology is composed of three main parts.
The first step is to extract relevant features from audio files, 
this is discussed on Section \vref*{feature-extraction}.\\
The next step consists in composing and refining a neural network 
to classify the data obtained from the previous step. This part is
discussed in Section \vref*{model-definition}.\\
Lastly, results from the classification, namely accuracy 
and standard deviation among test sets, are presented in 
Section \vref*{results}.

\paragraph{Implementation details}
The project is developed in \emph{Python} and some useful packages were used, 
in particular \emph{Librosa} and \emph{Dask} to extract features from audio files and 
\emph{TensorFlow} and \emph{Keras} to build the Neural Network.~\cite{python}~\cite{librosa}~\cite{dask}~\cite{tensorflow}~\cite{keras}
Other libraries were used to complete some minor tasks, they are named in 
the following Sections.

The project folder is structured as follows:
\begin{itemize}
    \item \emph{src}: this contains the source code for the
    project. Each sub-folder contains code for a specific part of the 
    processing. In particular, the \emph{data} folder holds classes to 
    extract features and to manage the dataset, the \emph{model} folder
    contains the class to create the Neural Network, then \emph{utils} 
    stores utility functions to measure performances;
    \item \emph{data}: here data is stored, there is a \emph{processed}
    and \emph{raw} sub-folders, where the first stores computed datasets 
    and the second original data;
    \item \emph{models}: trained models are saved here;
    \item \emph{notebooks}: the folder contains the Jupyter notebooks 
    used in the project, where code from src is practically used.
\end{itemize}
\newpage