\section{Feature extraction}

The dataset contains ten folds of audio samples, each one about 
four seconds long. The samples are divided in ten classes.

The training set consists of the first four folds plus the sixth, 
the other folds create five different test sets.

\subsection{First dataset}
Choosing the features to extract was difficult as I did not have 
prior experience working with audio. 

The \emph{Librosa} library provides many feature to choose from, 
for my first try with this dataset I kept it simple 
by opting for these three ones: 
\begin{enumerate}
    \item \emph{Mel-frequency cepstral coefficients}
    \item \emph{Chromagram}
    \item \emph{Root-mean-square}
\end{enumerate}
Each feature consists of an array of arrays containing measurements. 
I applied a series of functions to each sub-array and then concatenated 
the results in a final feature vector. 
The functions applied are \emph{minimum}, \emph{maximum}, \emph{mean} and \emph{median}.

This approach resulted in 132 components feature vectors.

\paragraph{Feature scaling}
After testing some neural networks on the first dataset the results 
were not promising. One of the reasons is the big difference in 
ranges among feature vector components.

\dots
% TALK ABOUT WHAT SCALER YOU USED

To mitigate this effect scaling was applied to the dataset.
This lead to an improvement on the results using the same 
model as before. 

The scaler trained on the training set was then 
used to scale the five different tests sets.

\subsection{Extended dataset}
To improve results on the test sets new features were added 
to the dataset, namely: 
\begin{enumerate}
    \item \emph{Zero-crossing rate}
    \item \emph{Roll-off frequency}
    \item \emph{Spectral flux onset strength}
\end{enumerate}
After applying the same four functions to these three new arrays, 
a total of 12 new features were added to the dataset.
Also, as scaling yield to promising results, 
the same scaler was used on this new dataset. 

After testing a really simple network on the new dataset 
results improved once again.

\paragraph{PCA}
Scaling lead to important improvements on the first dataset, 
so I decided to try the PCA to select the most important
features from the extended one.

\dots
% TALK ABOUT RESULTS